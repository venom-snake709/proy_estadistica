\documentclass[a4paper,10pt]{article}
\usepackage[utf8]{inputenc}
\usepackage[spanish]{babel}

\title{\textbf{Universidad Nacional Autónoma de México}\\ \textbf{Facultad de Estudios Superiores Acatlán}\\Estdística II: Análisis Histórico del Clima}

\date{\today}
\author{Alcantar Hernández Jessica Esmeralda\\Enríquez Sánchez Joshua Antonio\\Salgado Urtes Alexis\\Vilchis López Víctor Manuel}

\begin{document}
\maketitle
\newpage
\tableofcontents
\newpage

\begin{abstract}
    El presente trabajo es un estudio sobre el dataset \textit{Weather in Szeged 2006-2016}, el cual contiene datos históricos y notas sobre el clima en la ciudad de Szeged, Hungría.

    Mediante un análisis en Python, y usando los temas vistos en estadística II, se busca hacer un análisis científico sobre la relación de las variables y una regresión del sistema.
\end{abstract}
\newpage

\section{Resumen Ejecutivo}
\subsection{Tecnologías usadas}
Mediante el uso de Python3 y de las librerías \textbf{Pandas, StatsModels, SciPy y NumPy} estudiamos el dataset centrándonos en las predicción y el análisis de datos.

\subsection{Intervalos de confianza}
Como primer paso, se obtuvieron los intervalos de confianza al 95\%, esto es, el rango donde estarán los datos con un 95\% de confianza.

\subsection{Pruebas de hipótesis}
Continuamos con los análisis de las pruebas de hipótesis sobre los datos, entre las que se encuentran:
\begin{itemize}
 \item Prueba de Wilcoxon
 \item Prueba U Mann-Whitney
 \item Prueba Kruskal-Wallis
 \item Prueba Kendall
\end{itemize}
\subsection{Regresión Lineal}
Con los datos presentados, se hace un anális de regresión lineal sobre (agregar después)

\section{Tipos de Datos}
El dataset contiene unos 96453 registros, cada uno con 12 columnas. Los tipos de datos están distribuidos de la siguiente manera:
\begin{itemize}
 \item Formatted Date (Fecha con formato AAAA-MM-DD HH:MM:SS:MMM): Cualitativo
 \item Summary (Resumen): Cualitativo
 \item Precip Type (Tipo de precipitación): Cualitativo
 \item Daily Summary (Resumen diario): Cualitativo
 \item Loud Cover (Cobertura de nubes): Cualitativo
 \item Temperature °C (Temperatura): Cuantitativo
 \item Apparent Temperature °C (Temperatura aparente): Cuantitativo
 \item Humidity (Humedad): Cuantitativo
 \item Wind Speed (Velocidad del Viento): Cuantitativo
 \item Wind Bearing (Dirección del Viento): Cuantitativo
 \item Visibility (Visibilidad): Cuantitativo
 \item Pressure (Presión): Cuantitativa
\end{itemize}


\end{document}

